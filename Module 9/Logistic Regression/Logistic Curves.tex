\documentclass{article}
\usepackage{amsmath}

\begin{document}

\section{Logistic Regression and Logistic Curves}

Logistic regression is used to model the probability of a binary outcome based on one or more predictor variables. In the case of pine trees, we are interested in estimating the probability that a tree will produce cones based on its age.

\subsection{Estimating Probability}

Given a logistic regression model:

\[
\log \left( \frac{P(\text{Coning})}{1 - P(\text{Coning})} \right) = \beta_0 + \beta_1 \text{Age}
\]

where \( \beta_0 \) and \( \beta_1 \) are the coefficients estimated from the data, and \( P(\text{Coning}) \) is the probability of the tree producing cones. Rearranging to find \( P(\text{Coning}) \):

\[
P(\text{Coning}) = \frac{1}{1 + e^{-(\beta_0 + \beta_1 \text{Age})}}
\]

\textbf{Example:} Given \(\beta_0 = -2.9713\) and \(\beta_1 = 0.1961\), to find the probability of a 17-year-old tree producing cones:

\[
P(\text{Coning}) = \frac{1}{1 + e^{-(-2.9713 + 0.1961 \cdot 17)}}
\]

Calculate the exponent:

\[
\text{Exponent} = -(-2.9713 + 0.1961 \cdot 17) = 2.9713 - 3.3257 = -0.3544
\]

Thus:

\[
P(\text{Coning}) = \frac{1}{1 + e^{0.3544}} \approx \frac{1}{1 + 1.426} \approx \frac{1}{2.426} \approx 0.412
\]

The estimated probability of coning for a 17-year-old tree is approximately \textbf{0.412}.

\subsection{Interpreting the Logistic Slope}

The slope \( \beta_1 \) in the logistic regression model represents the change in the log odds of the outcome for a one-unit increase in the predictor variable. For a 1-year increase in age, the odds ratio is:

\[
\text{Odds Ratio} = e^{\beta_1}
\]

Given \(\beta_1 = 0.1961\):

\[
\text{Odds Ratio} = e^{0.1961} \approx 1.216
\]

This means that a tree that is 11 years old is approximately \textbf{1.22} times as likely to be producing cones as a tree that is 10 years old.

\subsection{Increasing Odds}

To find the odds ratio of coning for 12-year-old trees compared to 10-year-old trees:

\[
\text{Odds Ratio} = e^{\beta_1 \cdot (12 - 10)} = e^{0.1961 \cdot 2} = e^{0.3922} \approx 1.481
\]

The odds ratio of coning for 12-year-old trees compared to 10-year-old trees is approximately \textbf{1.48}.

\end{document}
