\documentclass{article}
\usepackage{amsmath}

\begin{document}

\section{Inferences for Logistic Regression}

In logistic regression, we assess the effect of predictor variables on the probability of a binary outcome by analyzing the log odds of the outcome.

\subsection{Testing the Hypothesis}

To determine if there is evidence that the rate of pine tree coning increases with age, we test the null hypothesis \( H_0: \beta_1 = 0 \) (no effect of age on coning). The alternative hypothesis is \( H_1: \beta_1 \neq 0 \) (age does affect coning).

Given:
\[
\hat{\beta}_1 = 0.19611
\]
\[
\text{Standard Error} = 0.07708
\]

The test statistic \( z \) is calculated as:
\[
z = \frac{\hat{\beta}_1}{\text{Standard Error}} = \frac{0.19611}{0.07708} \approx 2.544
\]

The one-sided p-value can be obtained using the normal distribution:
\[
\text{P-value} = \text{pnorm}(-2.544) \approx 0.0055
\]

Since the p-value is very small, we reject the null hypothesis. There is strong evidence that the rate of pine tree coning increases with age.

\subsection{Television and Criminal Convictions}

For the study on television watching and criminal convictions, we have the following logistic regression output:
\[
\text{Estimate for Television} = 0.4383
\]
\[
\text{Standard Error} = 0.1012
\]

\subsubsection{Hypothesis Test}

To determine the evidence of a positive association, calculate the z-value:
\[
z = \frac{0.4383}{0.1012} \approx 4.33
\]

Using the normal distribution:
\[
\text{P-value} = 1 - \text{pnorm}(4.33) \approx 0
\]

Since the p-value is very small, we have strong evidence of a positive association between the probability of a criminal conviction by age 26 and the average hours of television watched as a child.

\subsubsection{Confidence Intervals}

To compute a 95\% confidence interval for the increase in odds for a one-hour increase in television viewing:

\[
\text{Standard Error} = 0.1012
\]
\[
\text{Critical value} = 1.96 \text{ (for 95\% CI)}
\]

The confidence interval for \(\beta_1\) is:
\[
\text{CI} = \hat{\beta}_1 \pm (1.96 \times \text{Standard Error})
\]
\[
\text{CI} = 0.4383 \pm (1.96 \times 0.1012)
\]
\[
\text{CI} = 0.4383 \pm 0.1984
\]
\[
\text{Lower Bound} = 0.4383 - 0.1984 \approx 0.2399
\]
\[
\text{Upper Bound} = 0.4383 + 0.1984 \approx 0.6367
\]

Thus, the 95\% confidence interval for the increase in odds is \([0.2399, 0.6367]\).

\end{document}
