\documentclass{article}
\usepackage{amsmath}

\begin{document}

\section{Logistic Regression}

Logistic regression is used to model the probability of a binary outcome based on one or more predictor variables. It estimates the odds of an event occurring by applying the logistic function to a linear combination of the predictor variables.

\subsection{Estimating Odds}

In logistic regression, the model for the relationship between a predictor variable \( x \) and the binary outcome \( Y \) is given by:

\[
\log \left( \frac{P(Y = 1)}{1 - P(Y = 1)} \right) = \beta_0 + \beta_1 x
\]

where:
\begin{itemize}
    \item \( \frac{P(Y = 1)}{1 - P(Y = 1)} \) represents the odds of the outcome occurring.
    \item \( \beta_0 \) is the intercept.
    \item \( \beta_1 \) is the coefficient for the predictor variable \( x \).
\end{itemize}

The probability \( P(Y = 1) \) can be calculated using the logistic function:

\[
P(Y = 1) = \frac{1}{1 + e^{-(\beta_0 + \beta_1 x)}}
\]

To find the estimated odds for a given value of \( x \), use:

\[
\text{Odds} = e^{\beta_0 + \beta_1 x}
\]

\subsection{Example Calculation}

For a given logistic regression model, suppose the relationship between smoking status and the number of life-changing events is modeled as:

\[
\log \left( \frac{P(\text{Smoker})}{1 - P(\text{Smoker})} \right) = \beta_0 + \beta_1 (\text{LifeChangingEvents})
\]

To estimate the odds of being a smoker for someone with 9 life-changing events, substitute \( x = 9 \) into the formula:

\[
\text{Odds} = e^{\beta_0 + \beta_1 \cdot 9}
\]

You would need the values of \( \beta_0 \) and \( \beta_1 \) to perform this calculation. The resulting odds can then be compared with the given options.

\subsection{Equal Chance for the Response}

In logistic regression, the estimated probability of the response being 1 (i.e., the event occurring) is 0.5 (equal chance) when the linear combination of the predictors is zero. Therefore, the value of \( x \) that gives an equal chance for the response can be found by solving:

\[
\beta_0 + \beta_1 x = 0
\]

Thus, the value of \( x \) is:

\[
x = -\frac{\beta_0}{\beta_1}
\]

This value of \( x \) ensures that the odds ratio is 1, and thus the probability of the event occurring is 0.5.

\end{document}
