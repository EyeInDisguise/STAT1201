\documentclass{article}
\usepackage{amsmath}

\begin{document}

\title{Summary of Multiple Regression Analysis}
\author{}
\date{}
\maketitle

\section{Multiple Regression: Holding Breath}

Multiple regression allows us to study the effects of multiple explanatory variables on a response variable. An example from Module 8 examined the relationship between breath-holding time and height. Initially, the analysis might suggest a strong association between these variables. However, this association can be misleading if we do not account for other factors, such as sex.

\subsection{Initial Analysis}

The linear regression model analyzing breath-holding time with height gave the following results:
\begin{verbatim}
> breath = read.csv("Breath.csv")
> mod1 = lm(BreathHeld ~ Height, data=breath)
> summary(mod1)
\end{verbatim}
\begin{verbatim}
Coefficients:
             Estimate Std. Error t value Pr(>|t|)   
(Intercept) -181.7405    59.2889  -3.065  0.00667 **
Height         1.2774     0.3388   3.770  0.00140 **
\end{verbatim}

This model suggested a strong association between breath-holding time and height with an \( R^2 \) value of 0.4412.

\subsection{Inclusion of Categorical Variables}

The scatterplot analysis showed different relationships between breath-holding time and height for males and females, suggesting an interaction effect. To account for this, we include a dummy variable for sex and examine the interaction between sex and height:
\[
\text{BreathHeld} = \beta_0 + \beta_1 \text{Height} + \beta_2 \text{SexMale} + \beta_3 (\text{Height} \times \text{SexMale}) + \epsilon
\]

Fitting this model in R:
\begin{verbatim}
> mod2 = lm(BreathHeld ~ Sex*Height, data=breath)
> summary(mod2)
\end{verbatim}
\begin{verbatim}
Coefficients:
               Estimate Std. Error t value Pr(>|t|)
(Intercept)     84.3331    86.8828   0.971    0.346
SexMale        -41.7486   125.0206  -0.334    0.743
Height          -0.3468     0.5178  -0.670    0.513
SexMale:Height   0.4249     0.7158   0.594    0.561
\end{verbatim}

This analysis provided an \( R^2 \) value of 0.7847, but none of the variables were significant.

\subsection{Simplified Model}

Dropping the interaction term:
\[
\text{BreathHeld} = \beta_0 + \beta_1 \text{Height} + \beta_2 \text{SexMale} + \epsilon
\]

Fitting this simplified model in R:
\begin{verbatim}
> mod3 = lm(BreathHeld ~ Sex+Height, data=breath)
> summary(mod3)
\end{verbatim}
\begin{verbatim}
Coefficients:
            Estimate Std. Error t value Pr(>|t|)    
(Intercept)  47.0452    58.8693   0.799    0.435    
SexMale      32.3662     6.3267   5.116 8.61e-05 ***
Height       -0.1244     0.3506  -0.355    0.727    

Multiple R-squared:   0.78,	Adjusted R-squared:  0.7541 
\end{verbatim}

The simplified model shows strong evidence of a sex difference, but no evidence of a relationship between breath-holding time and height after accounting for sex differences. The \( R^2 \) value is 0.7800.

\section{Juicing Oranges}

To examine whether the variability in orange weight affects the results of the warming treatment, we use multiple regression including both orange weight and warming treatment:
\[
\text{JuiceAmount} = \beta_0 + \beta_1 \text{Weight} + \beta_2 \text{Treatment} + \epsilon
\]

We need to determine if there's evidence of an association with weight and warming treatment based on the results of this multiple regression.

\textbf{Possible outcomes}:
\begin{itemize}
    \item There is no evidence of an association with weight (\( p > 0.1 \)) and no evidence of a warming effect (\( p > 0.1 \)).
    \item There is no evidence of an association with weight (\( p > 0.1 \)) but strong evidence of a warming effect (\( p < 0.01 \)).
    \item There is weak evidence of an association with weight (\( 0.05 < p < 0.1 \)) but still no evidence of a warming effect (\( p > 0.1 \)).
    \item There is strong evidence of an association with weight (\( p < 0.01 \)) but still no evidence of a warming effect (\( p > 0.1 \)).
\end{itemize}

\end{document}
