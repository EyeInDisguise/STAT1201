\documentclass{article}
\usepackage{amsmath}
\usepackage{graphicx}

\begin{document}

\title{Summary of Two-Way ANOVA}
\author{}
\date{}
\maketitle

\section{One-Way ANOVA}

In a one-way ANOVA, we test whether there are significant differences in the mean of a response variable across multiple groups. The null hypothesis \( H_0 \) states that all group means are equal:

\[
H_0: \mu_1 = \mu_2 = \cdots = \mu_k
\]

Where \( k \) is the number of groups. The alternative hypothesis \( H_1 \) states that at least one group mean is different.

\subsection{Hypothesis Testing}

The strength of evidence against \( H_0 \) is determined by the p-value obtained from the ANOVA test:
\begin{itemize}
    \item **Strong evidence**: p-value < 0.01
    \item **Moderate evidence**: 0.01 < p-value < 0.05
    \item **Weak evidence**: 0.05 < p-value < 0.1
    \item **No evidence**: p-value > 0.1
\end{itemize}

\subsection{R-squared for One-Way ANOVA}

The \( R^2 \) value, or the proportion of variance explained by the model, is calculated as:

\[
R^2 = \frac{\text{Sum of Squares Between Groups}}{\text{Total Sum of Squares}}
\]

\section{Two-Way ANOVA}

Two-way ANOVA extends the one-way ANOVA to assess the impact of two factors on a response variable and their interaction.

\subsection{ANOVA Table}

Given:
\begin{itemize}
    \item Total Sum of Squares (SS): 357,387 lbs\(^2\)
    \item Phosphate Sum of Squares (SS): 164,872 lbs
    \item Nitrogen Sum of Squares (SS): 77,191 lbs
    \item Residual Sum of Squares (SS): 109,207 lbs
\end{itemize}

Degrees of Freedom (df):
\begin{itemize}
    \item Total df: \( N - 1 \)
    \item Phosphate df: \( \text{Number of levels of phosphate} - 1 \)
    \item Nitrogen df: \( \text{Number of levels of nitrogen} - 1 \)
    \item Residual df: Total df - Phosphate df - Nitrogen df - Interaction df
\end{itemize}

Mean Squares (MS):
\[
\text{MS}_{\text{Phosphate}} = \frac{\text{SS}_{\text{Phosphate}}}{\text{df}_{\text{Phosphate}}}
\]
\[
\text{MS}_{\text{Nitrogen}} = \frac{\text{SS}_{\text{Nitrogen}}}{\text{df}_{\text{Nitrogen}}}
\]
\[
\text{MS}_{\text{Residual}} = \frac{\text{SS}_{\text{Residual}}}{\text{df}_{\text{Residual}}}
\]

F-statistics:
\[
F_{\text{Phosphate}} = \frac{\text{MS}_{\text{Phosphate}}}{\text{MS}_{\text{Residual}}}
\]
\[
F_{\text{Nitrogen}} = \frac{\text{MS}_{\text{Nitrogen}}}{\text{MS}_{\text{Residual}}}
\]
\[
F_{\text{Interaction}} = \frac{\text{MS}_{\text{Interaction}}}{\text{MS}_{\text{Residual}}}
\]

P-values are determined from these F-statistics.

\subsection{Interaction Effect}

To test for an interaction effect between phosphate and nitrogen:
\begin{itemize}
    \item **Strong evidence of interaction**: p-value < 0.01
    \item **Moderate evidence**: 0.01 < p-value < 0.05
    \item **Weak evidence**: 0.05 < p-value < 0.1
    \item **No evidence**: p-value > 0.1
\end{itemize}

\subsection{Two-Way R-squared}

The \( R^2 \) value for the two-way ANOVA is calculated as:

\[
R^2 = \frac{\text{Sum of Squares Explained by Model}}{\text{Total Sum of Squares}}
\]

where the “Sum of Squares Explained by Model” includes SS for Phosphate, Nitrogen, and Interaction.

\end{document}
