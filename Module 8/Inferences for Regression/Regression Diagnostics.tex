\documentclass{article}
\usepackage{amsmath}
\usepackage{graphicx}

\begin{document}

\section{Regression Diagnostics}

In linear regression, it is crucial to check the following assumptions:
\begin{itemize}
    \item The underlying relationship between the variables is linear.
    \item The residuals (errors) are normally distributed.
    \item The residual variability is constant (homoscedasticity).
\end{itemize}

\subsection{Checking Linearity}

To check if the relationship is linear, plot the residuals against the explanatory variable. In R, you can do this as follows:

\begin{verbatim}
library(lattice)
xyplot(Residuals ~ Age, oxytocin, xlab="Age (years)", ylab="Residual (pg/mL)", type=c("p","smooth"))
\end{verbatim}

This plot should ideally show no obvious pattern. If a pattern is present, it may indicate a non-linear relationship.

\subsection{Checking Residual Variability}

You can also check for constant residual variability by plotting residuals against the fitted values:

\begin{verbatim}
xyplot(Residuals ~ predict(oxy.age), xlab="Fitted value (pg/mL)", ylab="Residual (pg/mL)", oxytocin, type=c("p","smooth"))
\end{verbatim}

This plot helps in assessing whether the variability of residuals remains constant across the range of fitted values.

\subsection{Checking Normality of Residuals}

To check if the residuals are normally distributed, you can use density and normal probability plots:

\begin{verbatim}
densityplot(~ Residuals, oxytocin, n=512, xlab="Residual (pg/mL)")
qqmath(~ Residuals, oxytocin, xlab="Z", ylab="Residual (pg/mL)")
\end{verbatim}

These plots help in determining if the residuals follow a normal distribution.

\section{Exponential Relationships}

Consider a model where the relationship between two variables is exponential, such as:

\[
y = ae^{-bt}
\]

To transform this into a linear relationship, take the natural logarithm of both sides:

\[
\ln(y) = \ln(a) - bt
\]

This gives a linear relationship between \(\ln(y)\) and \(t\). By fitting a least-squares line to \(\ln(y)\) versus \(t\), we can estimate \(a\) and \(b\). 

\subsection{Example: Exponential Fit}

Given the least-squares line in the transformed plot:

\[
\ln(y) = \text{intercept} + \text{slope} \cdot t
\]

Reversing the log-transform to find the exponential fit gives:

\[
y = e^{\text{intercept}} \cdot e^{(\text{slope} \cdot t)}
\]

\subsection{Estimating Exponential Fit}

To estimate the dissolving time \(y\) for a temperature \(t\), substitute the temperature into the exponential model. For instance, if the temperature is 50°C, calculate \(y\) based on the fitted exponential model.

\section{Power Relationships}

Power relationships can be modeled as:

\[
y = a x^b
\]

To linearize this, take the logarithm of both sides:

\[
\ln(y) = \ln(a) + b \ln(x)
\]

This transforms the power relationship into a linear one, which can be fitted using least-squares regression. From the fitted line, you can estimate \(a\) and \(b\).

\end{document}
