\documentclass{article}
\usepackage{amsmath}

\begin{document}

\section*{Summary}

In this section, we have learned that:

\begin{itemize}
    \item The least-squares line is the line that minimizes the sum of the squared prediction errors.
    \item Least-squares lines are sensitive to influential points, which can significantly affect the fit of the model.
    \item Regression analysis models the response variable as a linear function of the predictor variable, with residual variability around the mean response.
    \item Prediction errors for least-squares fitting are used to estimate the residual standard error, with degrees of freedom \( n - 2 \), where \( n \) is the number of observations.
    \item The standard inference in regression analysis involves testing whether the slope of the regression line is zero. A slope of zero suggests no association between the variables, while a significant nonzero slope indicates a potential association.
    \item A confidence interval estimates the mean response for a given predictor value, while a prediction interval estimates the range for an individual outcome.
    \item The assumptions of linear regression include: (1) the relationship between variables is linear, (2) the residuals are normally distributed, and (3) the residual variability is constant. These assumptions should be checked using residual plots.
    \item Certain nonlinear relationships can be transformed into linear relationships through appropriate transformations, such as logarithmic or power transformations.
\end{itemize}

\end{document}
