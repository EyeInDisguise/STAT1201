\documentclass{article}
\usepackage{amsmath}

\begin{document}

\section{Regression Analysis of Variance}

\subsection{Analysis of Variance (ANOVA)}

In regression analysis, ANOVA can be used to partition the total variability in the response variable into variability explained by the linear relationship and residual variability.

Given the least-squares regression line:

\[
\hat{y} = \beta_0 + \beta_1 x
\]

we decompose the total sum of squares (SS\(_\text{Total}\)) into:

\[
\text{SS}_\text{Total} = \text{SS}_\text{Regression} + \text{SS}_\text{Residual}
\]

where:

- \(\text{SS}_\text{Regression}\) measures variability explained by the model.
- \(\text{SS}_\text{Residual}\) measures unexplained variability.

For the oxytocin study, the ANOVA results are:

\begin{verbatim}
> oxy = read.csv("Oxytocin.csv")
> summary(aov(Basal ~ Age, data=oxy))
            Df Sum Sq Mean Sq F value  Pr(>F)   
Age          1 0.7777  0.7777   8.004 0.00977 **
Residuals   22 2.1377  0.0972
\end{verbatim}

The \( F \)-statistic is calculated as:

\[
F = \frac{\text{Mean Sq}_\text{Regression}}{\text{Mean Sq}_\text{Residual} }
\]

For this model:

\[
F = \frac{0.7777}{0.0972} = 8.004
\]

The \( p \)-value is \( 0.00977 \), which provides substantial evidence of an association between basal oxytocin levels and age.

\subsection{R-Squared}

The \( R^2 \) value represents the proportion of variance explained by the model. It is calculated as:

\[
R^2 = \frac{\text{SS}_\text{Regression}}{\text{SS}_\text{Total}}
\]

For the oxytocin study:

\[
R^2 = \frac{0.7777}{0.7777 + 2.1377} \approx 0.2668
\]

The \( R^2 \) value provides the proportion of total variability explained by the regression model. In this case, around 27% of the variability in basal oxytocin levels is explained by age.

\subsection{Pearson Correlation Coefficient}

The Pearson correlation coefficient \( r \) is related to \( R^2 \) by:

\[
R^2 = r^2
\]

Thus:

\[
r = \pm \sqrt{R^2}
\]

If \( R^2 = 0.391 \), then:

\[
r = \pm \sqrt{0.391} \approx \pm 0.625
\]

\subsection{p-Value Calculation}

The p-value for the \( F \)-statistic can be calculated using the cumulative distribution function for the \( F \)-distribution in R:

\[
\text{p-value} = 1 - \text{pf}(8.004, 1, 22)
\]

\subsection{Holding Breath Study}

To find the \( R^2 \) value for the relationship between time breath is held and height, we need to refer to the specific model output or provided options. Possible \( R^2 \) values could be:

\begin{itemize}
    \item 0.3388
    \item 0.4412
    \item 0.6642
\end{itemize}

\end{document}
