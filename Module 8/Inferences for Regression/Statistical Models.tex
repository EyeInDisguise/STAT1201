\documentclass{article}
\usepackage{amsmath}

\begin{document}

\section{Statistical Models}

\subsection{Linear Models}

In linear regression, the model is described by the equation:

\[
\hat{y} = \beta_0 + \beta_1 x
\]

where \( \hat{y} \) is the predicted value of the response variable, \( \beta_0 \) is the intercept, and \( \beta_1 \) is the slope of the line. The slope \( \beta_1 \) provides insight into the association between the response variable and the explanatory variable.

To test if there is evidence of an association between the variables, we test the null hypothesis:

\[
H_0: \beta_1 = 0
\]

against the alternative hypothesis:

\[
H_A: \beta_1 \neq 0
\]

In the oxytocin and emotions study, the least-squares fit for the relationship between basal plasma oxytocin level and age was found to be:

\[
\text{Basal} = 4.982 - 0.010 \times \text{Age}
\]

The model output from R provides the following:

\begin{verbatim}
> oxytocin = read.csv("Oxytocin.csv")
> oxy.age = lm(Basal ~ Age, data=oxytocin)
> summary(oxy.age)
\end{verbatim}

\begin{align*}
\text{Residual standard error} &= 0.3117 \text{ pg/mL} \\
\text{Intercept} &= 4.981984 \text{ (Estimate)} \\
\text{Slope} &= -0.009722 \text{ (Estimate)} \\
\text{Standard Error of Slope} &= 0.003436 \text{ pg/mL/year} \\
t &= \frac{\text{Estimate} - 0}{\text{Standard Error}} = \frac{-0.009722}{0.003436} \approx -2.829 \\
\text{P-value} &= 0.00977
\end{align*}

The p-value of 0.00977 indicates strong evidence that the slope is not zero, suggesting that plasma oxytocin level depends on age.

\subsection{Confidence Intervals}

A 95\% confidence interval for the slope is calculated as:

\[
\text{Estimate} \pm t_{\alpha/2} \times \text{Standard Error}
\]

Using the critical value \( t_{0.025} \approx 2.074 \), the interval is:

\[
-0.009722 \pm 2.074 \times 0.003436 = (-0.0168, -0.0026) \text{ pg/mL/year}
\]

This means we are 95\% confident that the mean oxytocin level of women decreases by between 0.0026 and 0.0168 pg/mL per year of age.

\section{Linear Model Questions}

\subsection{Breath Holding Study}

For a study where time breath can be held is related to height, if the p-value is:

\begin{itemize}
    \item \(0.131\) - No evidence of association.
    \item \(0.066\) - Weak evidence of association.
    \item \(0.027\) - Moderate evidence of association.
    \item \(0.0014\) - Strong evidence of association.
\end{itemize}

\subsection{Predictions}

To estimate the mean time breath can be held for people who are 170 cm tall, use the linear model. If the model predicts:

\begin{itemize}
    \item \(-181.74\) s
    \item \(1.277\) s
    \item \(35.42\) s
    \item \(217.16\) s
\end{itemize}

\end{document}
