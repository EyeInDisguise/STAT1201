\documentclass{article}
\usepackage{amsmath}

\begin{document}

\section{Confidence and Prediction Intervals}

\subsection{Confidence Intervals}

A confidence interval provides a range within which we are confident that a parameter, such as the mean response, lies. For a linear model, the 95\% confidence interval for the mean response at a specific value of the predictor \(x\) is calculated using the following steps:

1. **Estimate the Mean Response:**

   For a linear model \( y = \beta_0 + \beta_1 x \), the estimate of the mean response at \( x = x_0 \) is:

   \[
   \hat{y} = \hat{\beta_0} + \hat{\beta_1} x_0
   \]

   In R, this can be calculated using:

   \begin{verbatim}
   predict(model, newdata=data.frame(x=x_0), interval="confidence")
   \end{verbatim}

2. **Calculate the Confidence Interval:**

   The 95\% confidence interval is given by:

   \[
   \hat{y} \pm t_{\alpha/2, \text{df}} \cdot \text{SE}(\hat{y})
   \]

   where \( t_{\alpha/2, \text{df}} \) is the critical value from the \( t \)-distribution, and \( \text{SE}(\hat{y}) \) is the standard error of the estimate.

   For instance, if the R output is:

   \begin{verbatim}
   > predict(model, newdata=data.frame(Age=40), interval="confidence")
       fit     lwr      upr
   1 4.593114 4.46106 4.725168
   \end{verbatim}

   This indicates that we are 95\% confident that the mean basal oxytocin level for 40-year-old women is between 4.461 and 4.725 pg/mL.

\subsection{Prediction Intervals}

A prediction interval provides a range within which we are confident that a new individual observation will fall. It includes not only the uncertainty in the mean estimate but also the variability of individual responses.

1. **Estimate the Prediction Interval:**

   For predicting an individual response at \( x = x_0 \), the estimate is:

   \[
   \hat{y} = \hat{\beta_0} + \hat{\beta_1} x_0
   \]

   The prediction interval is given by:

   \[
   \hat{y} \pm t_{\alpha/2, \text{df}} \cdot \sqrt{\text{SE}(\hat{y})^2 + \text{Var}(\text{residuals})}
   \]

   where \( \text{Var}(\text{residuals}) \) is the variance of the residuals.

   In R, this can be calculated using:

   \begin{verbatim}
   predict(model, newdata=data.frame(x=x_0), interval="prediction")
   \end{verbatim}

   For example:

   \begin{verbatim}
   > predict(model, newdata=data.frame(Age=40), interval="prediction")
       fit      lwr      upr
   1 4.593114 3.933306 5.252921
   \end{verbatim}

   This indicates that the 95\% prediction interval for a 40-year-old woman’s basal oxytocin level ranges from 3.933 to 5.253 pg/mL.

\subsection{Examples}

\textbf{Breath Holding Study:}

For a linear model relating height and breath holding time, the 95\% confidence interval and prediction interval can be found similarly. Given the options:

\begin{itemize}
    \item 95\% Confidence Interval for height 170 cm: 
        \begin{itemize}
            \item (28.2, 42.7) s
            \item (29.4, 41.4) s
            \item (5.80, 65.0) s
        \end{itemize}
    \item 95\% Prediction Interval for height 170 cm: 
        \begin{itemize}
            \item (28.2, 42.7) s
            \item (29.4, 41.4) s
            \item (5.80, 65.0) s
        \end{itemize}
\end{itemize}

Select the appropriate interval based on your model output.

\end{document}
