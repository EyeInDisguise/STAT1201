\documentclass{article}
\usepackage{amsmath}

\begin{document}

\section{Multiple Testing}

When performing multiple hypothesis tests, the risk of Type I errors (false positives) increases. This is known as the problem of multiple comparisons. To control the overall error rate, adjustments to the \( p \)-values are necessary. This is crucial in various contexts, such as genetic studies where many tests are performed.

\subsection{Bonferroni Correction}

The Bonferroni correction is a widely used method to adjust \( p \)-values when multiple tests are conducted. It controls the family-wise error rate by adjusting the significance threshold for each individual test. 

If we are conducting \( m \) hypothesis tests and our target family-wise error rate is \( \alpha \), the significance threshold for each individual test is given by:

\[
\alpha_{\text{adjusted}} = \frac{\alpha}{m}
\]

where \( \alpha \) is the desired overall error rate, and \( m \) is the number of tests.

\subsection{Example Calculation}

Suppose we are conducting 20 hypothesis tests and our target error rate is \( \alpha = 0.05 \). Using the Bonferroni correction, the significance threshold for each individual test is calculated as follows:

\[
\alpha_{\text{adjusted}} = \frac{0.05}{20} = 0.0025
\]

Thus, each individual test should use a significance threshold of \( 0.0025 \) to control the overall error rate.

\end{document}
