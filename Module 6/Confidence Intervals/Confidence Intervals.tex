\documentclass{article}
\usepackage{amsmath}
\usepackage{amsfonts}
\usepackage{amssymb}

\begin{document}

\title{Confidence Interval for a Mean}
\author{}
\date{}
\maketitle

\section{Introduction}
A confidence interval (CI) for a population mean allows us to make statistical inferences about the population based on sample data. The CI provides a range within which the true population mean is likely to fall, with a specified level of confidence.

\section{Confidence Interval Formula}
To calculate a confidence interval for a population mean \(\mu\), use the following formula:

\[
\text{CI} = \bar{X} \pm z_{\alpha/2} \cdot \frac{\sigma}{\sqrt{n}}
\]

where:
\begin{itemize}
    \item \(\bar{X}\) is the sample mean.
    \item \(z_{\alpha/2}\) is the number of standard errors required for the desired confidence level, obtained from the standard normal distribution.
    \item \(\sigma\) is the population standard deviation (use \(s\) if \(\sigma\) is unknown).
    \item \(n\) is the sample size.
\end{itemize}

For a Normal distribution, \(95\%\) of outcomes occur within \(1.96\) standard deviations of the mean. For t-distributions, the critical value depends on the degrees of freedom and is found using the \(qt\) function in R.

\section{Example: Confidence Interval Calculation}
\subsection{Finding the Critical Value}
To calculate a \(95\%\) confidence interval using the t-distribution:
\begin{itemize}
    \item Determine the degrees of freedom (\(df = n - 1\)).
    \item Use the \(qt\) function in R to find the critical value:
    \[
    t_{\alpha/2, \text{df}} = \text{qt}(0.975, \text{df})
    \]
\end{itemize}

\subsection{Example Calculation}
Suppose you want to calculate a \(95\%\) confidence interval for a sample with \(n = 11\) observations:
\begin{itemize}
    \item Find \(t_{0.975, 10} \approx 2.262\) using R.
    \item Calculate the margin of error (ME) if \(\bar{X} = 50\) and \(s = 10\):
    \[
    \text{ME} = t_{\alpha/2, \text{df}} \cdot \frac{s}{\sqrt{n}} = 2.262 \cdot \frac{10}{\sqrt{11}} \approx 6.81
    \]
    \item Construct the CI:
    \[
    \text{CI} = \bar{X} \pm \text{ME} = 50 \pm 6.81 \text{ which gives } [43.19, 56.81]
    \]
\end{itemize}

\section{Confidence Interval for Mean Oxytocin Level}
\textbf{Data:}
\begin{itemize}
    \item Sample Mean (\(\bar{X}\)) = 4.429 pg/mL
    \item Sample Standard Deviation (\(s\)) = 0.2670 pg/mL
    \item Sample Size (\(n\)) = 12
\end{itemize}

\textbf{Calculate 95\% CI:}
\begin{itemize}
    \item Degrees of Freedom (\(df\)) = 12 - 1 = 11
    \item Critical Value: \(t_{0.975, 11} \approx 2.201\)
    \item Standard Error (SE):
    \[
    \text{SE} = \frac{s}{\sqrt{n}} = \frac{0.2670}{\sqrt{12}} \approx 0.077
    \]
    \item Margin of Error (ME):
    \[
    \text{ME} = t_{\alpha/2, \text{df}} \cdot \text{SE} = 2.201 \cdot 0.077 \approx 0.17
    \]
    \item Confidence Interval:
    \[
    \text{CI} = \bar{X} \pm \text{ME} = 4.429 \pm 0.17 = [4.259, 4.599]
    \]
\end{itemize}

\section{Confidence Interval for Mean Pulse Rate}
\textbf{Data:}
\begin{itemize}
    \item Sample Mean (\(\bar{X}\)) = 68.5 bpm
    \item Sample Standard Deviation (\(s\)) = 10.77 bpm
    \item Sample Size (\(n\)) = 60
\end{itemize}

\textbf{Calculate 95\% CI:}
\begin{itemize}
    \item Degrees of Freedom (\(df\)) = 60 - 1 = 59
    \item Critical Value: \(t_{0.975, 59} \approx 2.000\) (approximation for large \(n\))
    \item Standard Error (SE):
    \[
    \text{SE} = \frac{s}{\sqrt{n}} = \frac{10.77}{\sqrt{60}} \approx 1.39
    \]
    \item Margin of Error (ME):
    \[
    \text{ME} = t_{\alpha/2, \text{df}} \cdot \text{SE} = 2.000 \cdot 1.39 \approx 2.78
    \]
    \item Confidence Interval:
    \[
    \text{CI} = \bar{X} \pm \text{ME} = 68.5 \pm 2.78 = [65.7, 71.3]
    \]
\end{itemize}

\section{Conclusion}
Confidence intervals offer a range of plausible values for a population mean based on sample data. They help in understanding the precision of the sample estimate and the uncertainty associated with it.

\end{document}
