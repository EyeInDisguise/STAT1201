\documentclass{article}
\usepackage{amsmath}
\usepackage{graphicx}

\begin{document}

\section*{Sample Size and Margin of Error}

\subsection*{Choosing Sample Size}

To estimate the mean increase in pulse rate with a margin of error (\text{ME}) of 1.0 bpm at a 95\% confidence level, the required sample size can be determined. 

In an ideal situation where the population standard deviation (\(\sigma\)) is known, the margin of error is given by:

\[
\text{ME} = z_{\alpha/2} \cdot \frac{\sigma}{\sqrt{n}}
\]

Rearranging this formula to solve for the sample size \(n\):

\[
n = \left(\frac{z_{\alpha/2} \cdot \sigma}{\text{ME}}\right)^2
\]

For the caffeine example, suppose we have a pilot study that estimates \(\sigma = 5.587\) bpm and we desire a margin of error of 1.0 bpm at 95\% confidence. Thus:

\[
n = \left(\frac{1.96 \cdot 5.587}{1.0}\right)^2 \approx 120.24
\]

So, at least 120 subjects are needed to achieve the desired accuracy.

Given that \(\sigma\) and \(\text{ME}\) are fixed, the sample size \(n\) is proportional to the square of the reciprocal of the desired margin of error:

\[
n \propto \left(\frac{1}{\text{ME}}\right)^2
\]

Thus, to halve the margin of error, the sample size needs to be increased by a factor of four.

\subsection*{Confidence Level}

To improve precision by accepting a lower confidence level, we can use a smaller critical value. For instance, with 90\% confidence, the critical value is \(z_{0.95} \approx 1.833\). The spectrum of confidence intervals for different confidence levels is illustrated by:


\subsection*{Assumptions for T Methods}

When using the Student's \(t\)-distribution:
\begin{itemize}
    \item If the population is normally distributed, the sample mean will also be normally distributed.
    \item If the population is not normal, the Central Limit Theorem ensures that the sample mean is approximately normal as sample size increases.
\end{itemize}

Recommendations for sample size:
\begin{itemize}
    \item Small samples (\(n < 30\)): Use \(t\)-methods only if the data are symmetric and free of outliers.
    \item Moderate samples (\(30 \leq n < 100\)): \(t\)-methods are acceptable if there are no outliers or strong skewness.
    \item Large samples (\(n \geq 100\)): \(t\)-methods are generally acceptable, though outliers may still impact results.
\end{itemize}

\subsection*{Example: Coffee and Blood Flow}

Researchers measured myocardial blood flow (mL/min/g) before and after caffeine intake using PET scans for 8 volunteers. The data are:


To estimate the mean reduction, use the reductions data and apply statistical methods to determine the mean and the corresponding confidence interval.

\end{document}
