\documentclass{article}
\usepackage{amsmath}

\title{Summary: Sampling and Random Variables}
\author{}
\date{}

\begin{document}

\maketitle

\section{Summary}
In this section, we have learned the following key concepts:

\begin{itemize}
    \item \textbf{Probabilities and Sampling:} We use probabilities to describe the process of sampling from a population. This involves using probability models to make inferences about a population based on sample data.
    
    \item \textbf{Random Variables:} A random variable is a random process that results in a numerical outcome. It provides a way to quantify outcomes of random processes and is essential in statistical modeling.

    \item \textbf{Discrete Random Variables:} A discrete random variable has a countable set of possible outcomes. Probabilities for these outcomes can be assigned using a probability function, which defines the probability for each discrete outcome.

    \item \textbf{Continuous Random Variables:} For continuous random variables, the probability of any specific outcome is essentially 0. Instead, we calculate probabilities for intervals of outcomes. These probabilities are specified by a density curve, where the area under the curve over a given interval represents the probability of outcomes within that interval.
\end{itemize}

\end{document}
