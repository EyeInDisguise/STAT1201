\documentclass{article}
\usepackage{amsmath}

\begin{document}

\section*{Discrete Random Variables}

The \textbf{expected value} of a discrete random variable \(X\) is given by:
\[
E(X) = \sum_{i} x_i \cdot P(x_i)
\]
where \(x_i\) represents the possible values of the random variable and \(P(x_i)\) is the probability of \(x_i\).

This expected value is often called the \textbf{mean} of the random variable, denoted by \(\mu\):
\[
\mu = E(X)
\]

The \textbf{variance} of a random variable is the expected squared deviation from the mean:
\[
\text{Var}(X) = E[(X - \mu)^2]
\]

The \textbf{standard deviation} is the square root of the variance, bringing us back to the original units:
\[
\sigma = \sqrt{\text{Var}(X)}
\]

\section*{Law of Large Numbers}

The law of large numbers states that the sample mean of larger and larger samples gets closer and closer to the expected value of the random variable. 

A common misinterpretation is that after several identical outcomes (e.g., getting heads multiple times in coin tosses), the next outcome is more likely to balance the results. In reality, if the trials are independent, each outcome remains a 50-50 chance regardless of previous results. The law of large numbers pertains to the long-term average, not short-term fluctuations.

The phrase "law of large numbers" was introduced in 1837 by Siméon Poisson.

\section*{Continuous Random Variables}

For continuous random variables, we cannot directly sum probabilities because there are an uncountable number of outcomes. Instead, we use calculus and the concept of integration to handle these calculations.

Let \(X\) be a continuous random variable with probability density function (pdf) \(f(x)\). The expected value of \(X\) is given by:
\[
E(X) = \int_{-\infty}^{\infty} x \cdot f(x) \, dx
\]

The variance is:
\[
\text{Var}(X) = \int_{-\infty}^{\infty} (x - \mu)^2 \cdot f(x) \, dx
\]

where \(\mu = E(X)\). 

These calculations are often performed using software or calculators, especially for distributions like the Normal distribution.

\section*{Breathing Exercises}

Let \(W\) be the number of weeks of breathing exercises needed for improvement in a randomly selected patient, with the following probability function:

\[
\begin{array}{ccc}
x & 1 & 2 & 3 \\
P(x) & 0.34 & 0.42 & 0.24
\end{array}
\]

To find the \textbf{expected value} and \textbf{standard deviation} of \(W\):

\subsection*{Expected Value}

\[
E(W) = \sum_{i} x_i \cdot P(x_i)
\]
\[
E(W) = 1 \cdot 0.34 + 2 \cdot 0.42 + 3 \cdot 0.24
\]
\[
E(W) = 0.34 + 0.84 + 0.72 = 1.90
\]

\subsection*{Variance}

\[
\text{Var}(W) = \sum_{i} (x_i - E(W))^2 \cdot P(x_i)
\]
\[
\text{Var}(W) = (1 - 1.90)^2 \cdot 0.34 + (2 - 1.90)^2 \cdot 0.42 + (3 - 1.90)^2 \cdot 0.24
\]
\[
\text{Var}(W) = (-0.90)^2 \cdot 0.34 + (0.10)^2 \cdot 0.42 + (1.10)^2 \cdot 0.24
\]
\[
\text{Var}(W) = 0.81 \cdot 0.34 + 0.01 \cdot 0.42 + 1.21 \cdot 0.24
\]
\[
\text{Var}(W) = 0.2754 + 0.0042 + 0.2904 = 0.5700
\]

\subsection*{Standard Deviation}

\[
\sigma = \sqrt{\text{Var}(W)}
\]
\[
\sigma = \sqrt{0.5700} \approx 0.756
\]

\end{document}
