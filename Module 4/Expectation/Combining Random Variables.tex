\documentclass{article}
\usepackage{amsmath}

\begin{document}

\section*{Combining Random Variables}

\subsection*{Shifting and Scaling}

Consider a random variable \(X\) giving a temperature measurement in degrees Celsius with expected value \(E(X) = \mu_X\) and standard deviation \(\sigma_X\). If \(Y\) is a random variable giving the temperature in degrees Fahrenheit, then:
\[
Y = \frac{9}{5} X + 32
\]

The expected value and standard deviation of \(Y\) are given by:
\[
E(Y) = \frac{9}{5} E(X) + 32 = \frac{9}{5} \mu_X + 32
\]
\[
\sigma_Y = \frac{9}{5} \sigma_X
\]

Shifting by a constant \(a\) changes the expected value but does not affect the standard deviation:
\[
E(a + X) = a + E(X)
\]
\[
\sigma(a + X) = \sigma(X)
\]

Scaling by a constant \(b\) affects both the expected value and standard deviation:
\[
E(bX) = b E(X)
\]
\[
\sigma(bX) = |b| \sigma(X)
\]

\subsection*{Two Random Variables}

If \(X\) and \(Y\) are two random variables, their combined mean and variance are given by:
\[
E(X + Y) = E(X) + E(Y)
\]

For the variance:
\[
\text{Var}(X + Y) = \text{Var}(X) + \text{Var}(Y) + 2 \text{Cov}(X, Y)
\]

If \(X\) and \(Y\) are independent, then \(\text{Cov}(X, Y) = 0\) and:
\[
\text{Var}(X + Y) = \text{Var}(X) + \text{Var}(Y)
\]

The variance formula:
\[
\text{Var}(X - Y) = \text{Var}(X) + \text{Var}(Y)
\]

\subsection*{Daily Coffees}

Let \(X\) be the number of coffees purchased in a day by a random university student with the following probability function:

x = 0, 1, 2, 3
P(x) = 0.3, 0.4, 0.2, 0.1


\subsubsection*{Expected Value and Standard Deviation}

The expected value \(E(X)\) is:
\[
E(X) = \sum_{i} x_i \cdot P(x_i)
\]
\[
E(X) = 0 \cdot 0.3 + 1 \cdot 0.4 + 2 \cdot 0.2 + 3 \cdot 0.1
\]
\[
E(X) = 0 + 0.4 + 0.4 + 0.3 = 1.1
\]

The variance \(\text{Var}(X)\) is:
\[
\text{Var}(X) = \sum_{i} (x_i - E(X))^2 \cdot P(x_i)
\]
\[
\text{Var}(X) = (0 - 1.1)^2 \cdot 0.3 + (1 - 1.1)^2 \cdot 0.4 + (2 - 1.1)^2 \cdot 0.2 + (3 - 1.1)^2 \cdot 0.1
\]
\[
\text{Var}(X) = (-1.1)^2 \cdot 0.3 + (-0.1)^2 \cdot 0.4 + (0.9)^2 \cdot 0.2 + (1.9)^2 \cdot 0.1
\]
\[
\text{Var}(X) = 1.21 \cdot 0.3 + 0.01 \cdot 0.4 + 0.81 \cdot 0.2 + 3.61 \cdot 0.1
\]
\[
\text{Var}(X) = 0.363 + 0.004 + 0.162 + 0.361 = 0.89
\]

The standard deviation \(\sigma_X\) is:
\[
\sigma_X = \sqrt{\text{Var}(X)} = \sqrt{0.89} \approx 0.945
\]

\subsubsection*{Total Number of Coffees for 100 Students}

Let \(S\) be the total number of coffees purchased by 100 students. Then:
\[
E(S) = 100 \cdot E(X) = 100 \cdot 1.1 = 110
\]

\[
\text{Var}(S) = 100 \cdot \text{Var}(X) = 100 \cdot 0.89 = 89
\]

The standard deviation of \(S\) is:
\[
\sigma_S = \sqrt{\text{Var}(S)} = \sqrt{89} \approx 9.43
\]

\end{document}
