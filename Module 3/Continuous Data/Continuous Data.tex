\documentclass{article}
\usepackage{graphicx}

\title{Describing Distributions}
\author{}
\date{}

\begin{document}

\maketitle

\section*{Introduction}

In this module, we explore the variability present in a single quantitative variable. The pattern of variability is called the \textbf{distribution of the variable}, typically involving a \textbf{central tendency}, where observations gather around a central value with fewer observations further away.

We focus on three main aspects of the distribution:

\begin{itemize}
    \item The location or centre of the variability—a typical value taken by the variable.
    \item The spread of the variability—how far the values extend from the centre.
    \item The shape of the variability—whether values are spread symmetrically around the centre.
\end{itemize}

Once we describe the variability, we also look for patterns that don't match the general description, such as outliers or bimodal distributions.

\section*{Breath Holding Study}

A study asked 20 students to hold their breath for as long as physically comfortable. The time was recorded along with the sex and height of the students. All participants were non-smokers and were seated during the experiment. Below is a histogram showing the distribution of breath-holding times.

\subsection*{Breath Holding Times}

\noindent Based on the histogram, how would you describe the distribution of breath-holding times?

\begin{itemize}
    \item Symmetric
    \item Bimodal
    \item Skewed to the right
    \item Skewed to the left (correct)
\end{itemize}

\noindent We can also use a side-by-side dot plot to compare breath-holding times by sex:


\subsection*{Breath Holding Times by Sex}

How would you compare the distributions of breath-holding times between males and females?

\begin{itemize}
    \item The distributions have similar spread but males tend to hold their breath longer than females.
    \item The distributions have similar spread but females tend to hold their breath longer than males.
    \item Males have more variability than females in breath-holding times but are similar on average (correct).
    \item Females have more variability than males in breath-holding times but are similar on average.
\end{itemize}

\section*{Variability Patterns}

After describing variability using the aspects mentioned above, we look for patterns that deviate from this, such as outliers or bimodal distributions. For example, outliers are values that don't match the rest of the pattern, or we may encounter bimodal distributions, indicating two separate distributions.

These patterns will be discussed further in the module examples, starting with the following video.

\end{document}
