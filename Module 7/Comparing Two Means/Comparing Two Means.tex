\documentclass{article}
\usepackage{amsmath}
\usepackage{amssymb}
\usepackage{graphicx}

\begin{document}

\section{Comparing Two Means}

\subsection{Assumptions of the Welch Two-Sample T-Test}

The assumptions of the Welch two-sample t-test are:
\begin{itemize}
    \item \textbf{Independent groups:} The two groups being compared must be independent of each other.
    \item \textbf{Normal variability:} The data in each group should be approximately normally distributed.
    \item \textbf{Unequal variances:} The variances in the two groups do not need to be equal (this is what differentiates the Welch test from the pooled t-test).
\end{itemize}

\subsection{Pooled T-Test}

Suppose we are comparing two treatments with \( n_1 = 24 \) subjects in the first group and \( n_2 = 18 \) subjects in the second group. If the sample standard deviations seem similar, we can use a pooled t-test. 

The degrees of freedom (\( \text{df} \)) for the pooled t-test can be calculated using the formula:
\[
\text{df} = n_1 + n_2 - 2
\]
where \( n_1 \) and \( n_2 \) are the sample sizes of the two groups.

Substituting the values:
\[
\text{df} = 24 + 18 - 2 = 40
\]

Thus, the degrees of freedom for the pooled t-statistic are 40.

\end{document}
