\documentclass{article}
\usepackage{amsmath}
\usepackage{amssymb}

\begin{document}

\section*{Analysis of Variance: Green Tea Consumption}


\subsection*{ANOVA Table Setup}

A one-way analysis of variance (ANOVA) was performed to compare the mean body mass index (BMI) between the three green tea consumption groups. The given data are:

\[
\text{Group Sum of Squares (SS)} = 64.42
\]
\[
\text{Residual Sum of Squares (SS)} = 5809.64
\]

To complete the ANOVA table, we need to determine the degrees of freedom (DF) for each source of variation and the mean sum of squares (MS). 

\subsection*{Degrees of Freedom}

1. **Degrees of Freedom for Green Tea (\(\text{DF}_{\text{GreenTea}}\))**

   The degrees of freedom for the treatment (green tea groups) is calculated as:

   \[
   \text{DF}_{\text{GreenTea}} = k - 1
   \]

   where \( k \) is the number of groups. For three groups, \( k = 3 \), so:

   \[
   \text{DF}_{\text{GreenTea}} = 3 - 1 = 2
   \]

2. **Degrees of Freedom for Residuals (\(\text{DF}_{\text{Residuals}}\))**

   The total number of observations is \( N = 537 \). The degrees of freedom for residuals is:

   \[
   \text{DF}_{\text{Residuals}} = N - k = 537 - 3 = 534
   \]

3. **Degrees of Freedom for Total**

   The total degrees of freedom is:

   \[
   \text{DF}_{\text{Total}} = N - 1 = 537 - 1 = 536
   \]

   So the degrees of freedom for each source are:

   \[
   \begin{array}{|c|c|}
   \hline
   \text{Source} & \text{DF} \\
   \hline
   \text{GreenTea} & 2 \\
   \text{Residuals} & 534 \\
   \text{Total} & 536 \\
   \hline
   \end{array}
   \]

\subsection*{Mean Sum of Squares}

The mean sum of squares (MS) for each source is calculated as follows:

1. **Mean Sum of Squares for Green Tea (\(\text{MS}_{\text{GreenTea}}\))**

   \[
   \text{MS}_{\text{GreenTea}} = \frac{\text{SS}_{\text{GreenTea}}}{\text{DF}_{\text{GreenTea}}} = \frac{64.42}{2} = 32.21
   \]

2. **Mean Sum of Squares for Residuals (\(\text{MS}_{\text{Residuals}}\))**

   \[
   \text{MS}_{\text{Residuals}} = \frac{\text{SS}_{\text{Residuals}}}{\text{DF}_{\text{Residuals}}} = \frac{5809.64}{534} \approx 10.87
   \]

\subsection*{F-Statistic}

The F-statistic is calculated as:

\[
F = \frac{\text{MS}_{\text{GreenTea}}}{\text{MS}_{\text{Residuals}}} = \frac{32.21}{10.87} \approx 2.96
\]

\subsection*{Conclusion}

To determine the significance, compare the calculated F-value with the critical value from the F-distribution table at a chosen significance level (typically \(\alpha = 0.05\)). Based on the F-value and p-value associated with it:

\begin{itemize}
    \item No evidence of a difference in mean BMI between the three green tea consumption groups (p = 0.151).
    \item Weak evidence of a difference in mean BMI between the three green tea consumption groups (p = 0.053).
    \item Moderate evidence of a difference in mean BMI between the three green tea consumption groups (p = 0.0377).
    \item Strong evidence of a difference in mean BMI between the three green tea consumption groups (p = 0.0082).
\end{itemize}

Based on the F-statistic, if the calculated p-value is less than the significance level, there is evidence of a difference in BMI between the groups. 

\end{document}
