\documentclass{article}
\usepackage{amsmath}
\usepackage{amssymb}

\begin{document}

\section*{Assumptions for One-Way ANOVA}

In one-way analysis of variance (ANOVA), we represent the model as follows:

\[
Y_{ij} = \mu_i + \epsilon_{ij}
\]

where:
\begin{itemize}
    \item \(Y_{ij}\) is the observation for the \(j\)-th subject in group \(i\),
    \item \(\mu_i\) is the mean response for the group \(i\),
    \item \(\epsilon_{ij}\) represents the residuals or errors, assumed to be normally distributed with mean 0 and constant variance \(\sigma^2\).
\end{itemize}

The model assumes:
\begin{enumerate}
    \item \textbf{Groups are independent:} The observations in each group are independent of each other.
    \item \textbf{Residuals have a Normal distribution:} The residuals \(\epsilon_{ij}\) are normally distributed.
    \item \textbf{Homogeneity of variances:} The variability (variance) of the residuals is constant across all groups.
\end{enumerate}

These assumptions can be checked using:
\begin{itemize}
    \item Side-by-side box plots of the original values.
    \item Plots of the residuals (the differences between observed values and the sample means for their group).
\end{itemize}

\section*{Rate of Dissolution}

A student experiment investigated the effect of water temperature on the dissolution times of soluble aspirin tablets. Three temperature conditions were tested:
\begin{itemize}
    \item Hot (78-90℃),
    \item Cold (7-10℃),
    \item Tap water (15-17℃).
\end{itemize}
A total of 60 tablets were used, with 20 tablets for each temperature condition. The dissolving times (in seconds) are recorded in the file \texttt{Tablets.csv}.

\subsection*{Dealing with Variability}

Initial analysis showed that while the distributions were fairly symmetric, they had different standard deviations. This pattern is common, where larger magnitude values exhibit greater variability.

\subsubsection*{Transformation Techniques}

To address the issue of different standard deviations, several transformations were considered:
\begin{itemize}
    \item \textbf{Log Transformation:} Applied to transform a multiplicative effect into an additive one. However, after log transformation, variability between groups remained large.
    \item \textbf{Reciprocal Transformation:} The reciprocal of each observation was taken (e.g., a measurement of 98 seconds becomes \(1/98 = 0.0102\) seconds). This transformation stabilizes the variability and converts the measurement to a rate of dissolution (e.g., 0.0102 tablets per second).
\end{itemize}

The reciprocal transformation was successful in stabilizing variability, and the transformed values can be used for further analysis of variance.

\end{document}
