\documentclass{article}
\usepackage{amsmath}

\title{Summary of ANOVA and Post-Hoc Tests}
\author{}
\date{}

\begin{document}

\maketitle

\section{Summary}

In this section we have learned that:

\begin{itemize}
    \item Analysis of variance (ANOVA) breaks down the total variability in a response into the residual variability and the variability that can be explained by the model.
    \item Each variability component is summarized by a sum of squared deviations and a degrees of freedom.
    \item The \( R^2 \) value gives the proportion of the total sum of squared deviations that can be explained by the model.
    \item The \( F \) statistic is used in conjunction with the \( F \) distribution to test whether the variability due to the model is significantly more than the residual variability.
    \item Making multiple comparisons leads to an inflation of the family error rate, which needs to be corrected.
    \item The Bonferroni correction adjusts the individual error rate to satisfy a target family error rate, such as 5\%.
    \item Tukey’s HSD is a popular method for generating confidence intervals and adjusted \( p \)-values for comparing group means.
\end{itemize}

\end{document}
