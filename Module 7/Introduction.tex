\documentclass{article}
\usepackage{amsmath}
\usepackage{amsfonts}
\usepackage{amssymb}

\title{Module 7: Advanced Statistical Techniques}
\author{}
\date{}

\begin{document}

\maketitle

\section{Comparing Two Groups with the t Distribution}

\subsection{Two-Sample t-Test}
When comparing the means of two independent groups, we use the two-sample t-test. The test statistic is given by:
\begin{equation}
t = \frac{\bar{X}_1 - \bar{X}_2}{\sqrt{s_p^2 \left(\frac{1}{n_1} + \frac{1}{n_2}\right)}}
\end{equation}
where $\bar{X}_1$ and $\bar{X}_2$ are the sample means, $s_p^2$ is the pooled variance, and $n_1$ and $n_2$ are the sample sizes.

\subsection{Pooled Estimate of Standard Deviation}
When the variances of the two groups are assumed to be equal, the pooled estimate of standard deviation $s_p$ is calculated as:
\begin{equation}
s_p^2 = \frac{(n_1 - 1)s_1^2 + (n_2 - 1)s_2^2}{n_1 + n_2 - 2}
\end{equation}
where $s_1^2$ and $s_2^2$ are the sample variances of the two groups.

\section{Determining Optimum Group Sizes}

\subsection{Sample Size Calculation}
To achieve a desired power for detecting a difference between two groups, we calculate the required sample size using:
\begin{equation}
n = \left(\frac{Z_{\alpha/2} + Z_{\beta}}{\delta / \sigma}\right)^2
\end{equation}
where $Z_{\alpha/2}$ is the critical value for the desired significance level, $Z_{\beta}$ is the critical value for the desired power, $\delta$ is the effect size, and $\sigma$ is the standard deviation.

\section{Analysis of Variance (ANOVA)}

\subsection{One-Way ANOVA}
One-way ANOVA is used to compare means across three or more groups. The test statistic is:
\begin{equation}
F = \frac{\text{Between-group variance}}{\text{Within-group variance}}
\end{equation}
where the between-group variance measures the variance among group means, and the within-group variance measures the variance within each group.

\subsection{Reasoning Behind ANOVA}
ANOVA is used to test the hypothesis that the means of several groups are equal. It extends the idea of the t-test to more than two groups by comparing variances.

\section{Multiple Hypothesis Testing}

\subsection{Issues with Multiple Comparisons}
When performing multiple hypothesis tests, the risk of Type I errors increases. Multiple testing increases the chance of finding at least one false positive.

\subsection{Bonferroni Correction}
To control the overall Type I error rate, the Bonferroni correction adjusts the significance level:
\begin{equation}
\alpha_{\text{adj}} = \frac{\alpha}{m}
\end{equation}
where $\alpha$ is the original significance level and $m$ is the number of tests.

\end{document}
