\documentclass{article}
\usepackage{amsmath}

\begin{document}

\title{Sample Counts and Proportions Expected Values}
\author{}
\date{}
\maketitle

\section{Sample Count}
Let \( X \) be the count of successes from \( n \) independent Bernoulli trials, each with success probability \( p \). To find the expected value and standard deviation of \( X \), we define new random variables \( X_i \) for each of the Bernoulli trials:

\[
X_i =
\begin{cases}
1 & \text{if success} \\
0 & \text{if failure}
\end{cases}
\]

The probability function for each \( X_i \) is given by:
\[
P(X_i = 1) = p, \quad P(X_i = 0) = 1 - p
\]

\subsection{Expected Value and Standard Deviation}
The expected value of \( X_i \) is:
\[
E[X_i] = 1 \cdot p + 0 \cdot (1 - p) = p
\]

The variance of \( X_i \) is:
\[
\text{Var}(X_i) = E[X_i^2] - (E[X_i])^2 = p - p^2 = p(1 - p)
\]

The count of successes \( X \) is:
\[
X = \sum_{i=1}^n X_i
\]

The expected value of \( X \) is:
\[
E[X] = E\left[\sum_{i=1}^n X_i\right] = \sum_{i=1}^n E[X_i] = np
\]

The variance of \( X \) is:
\[
\text{Var}(X) = \text{Var}\left(\sum_{i=1}^n X_i\right) = \sum_{i=1}^n \text{Var}(X_i) = np(1 - p)
\]

The standard deviation of \( X \) is:
\[
\sigma_X = \sqrt{\text{Var}(X)} = \sqrt{np(1 - p)}
\]

\section{Sample Proportion}
The sample proportion \( \hat{p} \) is defined as:
\[
\hat{p} = \frac{X}{n}
\]

The expected value of \( \hat{p} \) is:
\[
E[\hat{p}] = E\left[\frac{X}{n}\right] = \frac{E[X]}{n} = p
\]

The variance of \( \hat{p} \) is:
\[
\text{Var}(\hat{p}) = \text{Var}\left(\frac{X}{n}\right) = \frac{\text{Var}(X)}{n^2} = \frac{p(1 - p)}{n}
\]

The standard deviation of \( \hat{p} \) is:
\[
\sigma_{\hat{p}} = \sqrt{\text{Var}(\hat{p})} = \sqrt{\frac{p(1 - p)}{n}}
\]

\section{Application Problems}

\subsection{Expected Values and Standard Deviations}
\textbf{Example: Barred Spiral Galaxies}

Suppose a new survey of 42 spiral galaxies will be conducted, with the null hypothesis that 30\% of spiral galaxies are barred.

\begin{itemize}
    \item \textbf{Sample Count:}
    \[
    E[X] = np = 42 \times 0.3 = 12.6
    \]
    \[
    \sigma_X = \sqrt{np(1 - p)} = \sqrt{42 \times 0.3 \times 0.7} \approx 3.4
    \]

    \item \textbf{Sample Proportion:}
    \[
    E[\hat{p}] = p = 0.3
    \]
    \[
    \sigma_{\hat{p}} = \sqrt{\frac{p(1 - p)}{n}} = \sqrt{\frac{0.3 \times 0.7}{42}} \approx 0.085
    \]
\end{itemize}

\subsection{Sample Size for Desired Standard Deviation}
To find the smallest sample size \( n \) that ensures the standard deviation of the sample proportion is no more than 2\% (0.02), we use:
\[
\sigma_{\hat{p}} = \sqrt{\frac{p(1 - p)}{n}} \leq 0.02
\]
\[
\frac{p(1 - p)}{n} \leq 0.02^2
\]
\[
n \geq \frac{p(1 - p)}{0.02^2}
\]
For \( p = 0.3 \):
\[
n \geq \frac{0.3 \times 0.7}{0.02^2} \approx 525
\]

\end{document}
